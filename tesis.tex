\documentclass[oneside,numbers,spanish,a4paper]{ezthesis/ezthesis}
%% # Opciones disponibles para el documento #
%%
%% Las opciones con un (*) son las opciones predeterminadas.
%%
%% Modo de compilar:
%%   draft            - borrador con marcas de fecha y sin im'agenes
%%   draftmarks       - borrador con marcas de fecha y con im'agenes
%%   final (*)        - version final de la tesis
%%
%% Tama'no de papel:
%%   letterpaper (*)  - tama'no carta (Am'erica)
%%   a4paper          - tama'no A4    (Europa)
%%
%% Formato de impresi'on:
%%   oneside          - hojas impresas por un solo lado
%%   twoside (*)      - hojas impresas por ambos lados
%%
%% Tama'no de letra:
%%   10pt, 11pt, o 12pt (*)
%%
%% Espaciado entre renglones:
%%   singlespace      - espacio sencillo
%%   onehalfspace (*) - espacio de 1.5
%%   doublespace      - a doble espacio
%%
%% Formato de las referencias bibliogr'aficas:
%%   numbers          - numeradas, p.e. [1]
%%   authoryear (*)   - por autor y a'no, p.e. (Newton, 1997)
%%
%% Opciones adicionales:
%%   spanish         - tesis escrita en espa'nol
%%
%% Desactivar opciones especiales:
%%   nobibtoc   - no incluir la bibiolgraf'ia en el 'Indice general
%%   nofancyhdr - no incluir "fancyhdr" para producir los encabezados
%%   nocolors   - no incluir "xcolor" para producir ligas con colores
%%   nographicx - no incluir "graphicx" para insertar gr'aficos
%%   nonatbib   - no incluir "natbib" para administrar la bibliograf'ia

%% Paquetes adicionales requeridos se pueden agregar tambi'en aqu'i.
%% Por ejemplo:
%\usepackage{subfig}
%\usepackage{multirow}
\usepackage[utf8]{inputenc}
\usepackage{mathptmx} % Times new roman
\usepackage{enumitem}
\usepackage[export]{adjustbox}
\usepackage[figurename=Fig.,tablename=TABLA]{caption}
\usepackage{float}
\usepackage{flafter}
\usepackage{afterpage}
\usepackage{placeins}
\usepackage{tabularx}
\usepackage{chngcntr}
\usepackage{titlesec}
\usepackage{lscape}
\usepackage{listings} % code
\usepackage{color}
\usepackage{textcomp} % quotesimple
\usepackage{multicol}

\definecolor{codegreen}{rgb}{0,0.6,0}
\definecolor{codegray}{rgb}{0.5,0.5,0.5}
\definecolor{codepurple}{rgb}{0.58,0,0.82}
\definecolor{backcolour}{rgb}{0.95,0.95,0.92}
 
\lstdefinestyle{codestyle}{
    backgroundcolor=\color{backcolour},
    commentstyle=\color{codegreen},
    keywordstyle=\color{magenta},
    numberstyle=\tiny\color{codegray},
    stringstyle=\color{codepurple},
    basicstyle=\footnotesize,
    breakatwhitespace=false,
    breaklines=true,
    captionpos=b,
    keepspaces=false,
    numbers=left,
    numbersep=5pt,
    showspaces=false,
    showstringspaces=false,
    showtabs=false,
    tabsize=2,
    inputencoding=utf8,
    literate={á}{{\'a}}1 {é}{{\'e}}1 {í}{{\'i}}1 {ó}{{\'o}}1 {ú}{{\'u}}1
}
\lstset{style=codestyle}

%% # Datos del documento #
%% Nota que los acentos se deben escribir: \'a, \'e, \'i, etc.
%% La letra n con tilde es: \~n.

\author{Kelvin Helmut Provincia Quispe}
\title{PLANTILLA DE PROYECTO DE FIN DE CARRERA}
\degree{[Grado o Título a obtener]}
\supervisor{Nombre de mi Asesor}
\institution{Universidad La Salle}
\faculty{ESCUELA PROFESIONAL DE INGENIERÍA DE SOFTWARE}
\department{AREQUIPA – PERÚ}

%% # M'argenes del documento #
%% 
%% Quitar el comentario en la siguiente linea para ajustar los m'argenes del
%% documento. Leer la documentaci'on de "geometry" para m'as informaci'on.

%\geometry{top=40mm,bottom=33mm,inner=40mm,outer=25mm}

%% El siguiente comando agrega ligas activas en el documento para las
%% referencias cruzadas y citas bibliogr'aficas. Tiene que ser *la 'ultima*
%% instrucci'on antes de \begin{document}.

\hyperlinking
\begin{document}

%% En esta secci'on se describe la estructura del documento de la tesis.
%% Consulta los reglamentos de tu universidad para determinar el orden
%% y la cantidad de secciones que debes de incluir.


\counterwithout{figure}{chapter} % chngcntr
\counterwithout{table}{chapter}

\renewcommand\thechapter{\Roman{chapter}}
\renewcommand{\thesubsection}{\Alph{subsection}}
\renewcommand{\thesection}{\arabic{chapter}.\arabic{section}}
\renewcommand\listtablename{Índice de tablas}
\renewcommand\bibname{Referencias}
\renewcommand\thefigure{\arabic{figure}}
\renewcommand\thetable{\Roman{table}}

\titleformat{\chapter}[hang] {\vspace{-6em}\centering\normalfont\LARGE\bfseries}{\chaptertitlename\ \thechapter\ --}{0.2em}{}

\captionsetup[table]{
  labelsep=newline,
  justification=centering,
}

\fancyfoot[R]{\thepage}
\fancyhead[R]{}

%% # Portada de la tesis #
%% Mirar el archivo "titlepage.tex" para los detalles.
\include{titlepage}

%% # Prefacios #
%% Por cada prefacio (p.e. agradecimientos, resumen, etc.) crear
%% un nuevo archivo e incluirlo aqu'i.
%% Para m'as detalles y un ejemplo mirar el archivo "gracias.tex".
% \prefacesection{Dedicatoria}
\thispagestyle{empty}\mbox{}
\vspace*{4cm}
\begin{flushright}
    \emph{Dedicatoria}
\end{flushright}
\include{prefacios/agradecimiento}

%% # 'Indices y listas de contenido #
%% Quitar los comentarios en las lineas siguientes para obtener listas de
%% figuras y cuadros/tablas.
\tableofcontents
\prefacesection{Índice de abreviaturas y siglas}

\listoftables
\listoffigures
\prefacesection{Glosario de términos}

\prefacesection{Resumen}
Resumen\cite{NewCam97}

\include{prefacios/abstract}
\include{prefacios/palabras_clave}

%% # Cap'itulos #
%% Por cada cap'itulo hay que crear un nuevo archivo e incluirlo aqu'i.
%% Mirar el archivo "intro.tex" para un ejemplo y recomendaciones para
%% escribir.
\chapter{Problemática del Proyecto}

\section{Contexto del problema}

\section{Antecedentes}

\chapter{Planteamiento del Proyecto}
\section{Fundamentos teóricos}

\section{Objetivos del Proyecto}
\subsection{Objetivo General}

\subsection{Objetivos Específicos}

\section{Justificación}

\section{Viabilidad}
\subsection{Viabilidad técnica}

\subsection{Viabilidad operativa}

\subsection{Viabilidad económica}

\section{Limitaciones}

\include{capitulos/03_metodologia}
%\include{conclu}

\appendix
%% Cap'itulos incluidos despues del comando \appendix aparecen como ap'endices
%% de la tesis.
%\include{apendiceA}
%\include{apendiceB}
%\include{apendiceC}

%% Incluir la bibliograf'ia. Mirar el archivo "biblio.bib" para m'as detales
%% y un ejemplo.
% Ver ezthesis.cls linea 178
%\bibliographystyle{IEEEannot}
\bibliography{bibliografia/biblio}

\end{document}
